\section{\texttt{rgl}}

\subsection{Descrição}

%----------------------------------------------------------------------

\begin{frame}

  \texttt{rgl} oferece funções de médio a alto nível para gráficos 3D
  interativos, tanto as representações em 3D de gráficos simples em 2D
  quanto para representações objetos geométricos no espaço (cubos,
  elipses, etc). A visualização pode ser na tela com OpenGL ou em outros
  formados como WebGL.

  \begin{itemize}
  \item Autores: Daniel Adler, Duncan Murdoch, e outros
  \item Lançamento: 04-Mar-2004
  \item Versão: 0.95.1247
  \item URL: \url{http://cran.r-project.org/web/packages/rgl/index.html}
  \item Algumas aplicações com o \texttt{rgl}:

  \begin{itemize}
    \itemsep1pt\parskip0pt\parsep0pt
  \item \href{http://www.r-bloggers.com/?s=rgl}{Busca no R Bloggers}
  \end{itemize}
\end{itemize}

\end{frame}

%----------------------------------------------------------------------

\subsection{Como usar}

\begin{frame}

  modelo característico.

\end{frame}

%----------------------------------------------------------------------

\subsection{Exemplos}

\begin{frame}

  links para executáveis e galerias.

\end{frame}