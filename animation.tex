\section{\texttt{animation}}

%----------------------------------------------------------------------

\subsection{Descrição}

\begin{frame}

  \begin{quote}
    To turn ideas in animations (as quick and faithfully as possible).
  \end{quote}

  \texttt{animation} contém funções para produzir
  animações com o R. As animações podem ser produzidas em vários
  formatos: flash, gif, html, pdf e vídeos.

  \begin{itemize}
  \item Autores: Yihui Xie, Lijia Yu, Weicheng Zhu.
  \item Lançamento: 11-Nov-2007.
  \item Versão: 2.3.
  \item URL:
    \url{http://cran.r-project.org/web/packages/animation/index.html},
    \url{http://yihui.name/animation/}
  \item Third-party software: ImageMagik (gif, mpeg convert), SWF Tools
    (\texttt{png2swf}, \texttt{jpeg2swf}, \texttt{pdf2swf})
  \end{itemize}

\end{frame}

%----------------------------------------------------------------------

\begin{frame}
  \begin{itemize}
  \item Na janela gráfica
    \begin{itemize}
    \item Mais natural;
    \item Não requer software extra.
    \end{itemize}
  \item HTML
    \begin{itemize}
    \item Não requer software extra, apenas navegador;
    \item Interface de um player de vídeo com botões de play, pause,
      etc;
    \item Não precisa ter o R, pode usar o Rweb.
    \end{itemize}
  \item GIF
    \begin{itemize}
    \item Requer \texttt{ImageMagick} ou \texttt{GraphicsMagick} para
      converter sequência de imagens em gifs.
    \end{itemize}
  \item Video
    \begin{itemize}
    \item Requer \texttt{FFmpeg} para converter sequência de imagens em
      vídeos.
    \end{itemize}
  \item Flash
    \begin{itemize}
    \item Requer \texttt{SWFTools} para criar animações em flash.
    \end{itemize}
  \end{itemize}
\end{frame}

%----------------------------------------------------------------------

\subsection{Como usar}

\begin{frame}

\vspace{-0.5cm}
\begin{knitrout}\footnotesize
\definecolor{shadecolor}{rgb}{0.969, 0.969, 0.969}\color{fgcolor}\begin{kframe}
\begin{alltt}
\hlkwd{require}\hlstd{(animation)}

\hlstd{x} \hlkwb{<-} \hlstd{precip}
\hlstd{a} \hlkwb{<-} \hlkwd{extendrange}\hlstd{(x)}

\hlkwd{ani.options}\hlstd{(}\hlkwc{interval} \hlstd{=} \hlnum{0.3}\hlstd{)}
\hlkwa{for}\hlstd{(i} \hlkwa{in} \hlnum{1}\hlopt{:}\hlnum{30}\hlstd{)\{}
    \hlstd{bks} \hlkwb{<-} \hlkwd{seq}\hlstd{(a[}\hlnum{1}\hlstd{], a[}\hlnum{2}\hlstd{],} \hlkwc{length.out} \hlstd{= i} \hlopt{+} \hlnum{1}\hlstd{)}
    \hlkwd{hist}\hlstd{(x,} \hlkwc{breaks} \hlstd{= bks)}
    \hlkwd{ani.pause}\hlstd{()}
\hlstd{\}}
\end{alltt}
\end{kframe}
\end{knitrout}
\pause

\vspace{-0.4cm}
\begin{knitrout}\footnotesize
\definecolor{shadecolor}{rgb}{0.969, 0.969, 0.969}\color{fgcolor}\begin{kframe}
\begin{alltt}
\hlkwd{saveGIF}\hlstd{(\{}
    \hlkwa{for}\hlstd{(i} \hlkwa{in} \hlnum{1}\hlopt{:}\hlnum{30}\hlstd{)\{}
        \hlstd{bks} \hlkwb{<-} \hlkwd{seq}\hlstd{(a[}\hlnum{1}\hlstd{], a[}\hlnum{2}\hlstd{],}
                   \hlkwc{length.out} \hlstd{= i} \hlopt{+} \hlnum{1}\hlstd{)}
        \hlkwd{hist}\hlstd{(x,} \hlkwc{breaks} \hlstd{= bks)}
        \hlkwd{ani.pause}\hlstd{()}
    \hlstd{\}}
\hlstd{\},} \hlkwc{interval} \hlstd{=} \hlnum{0.3}\hlstd{)}
\end{alltt}
\end{kframe}
\end{knitrout}


\end{frame}

%----------------------------------------------------------------------

\subsection{Exemplos}

\begin{frame}

  Praticando:
  \begin{enumerate}
  \item \href{run:./R/animation/animation.R}{R Script animation}
  \item \href{run:./animation/animation.html}{Galeria animation IGUIR}
  \end{enumerate}
  
  \vspace{0.5cm}
  Algumas aplicações com o animation:
  \begin{itemize}
  \item \href{http://vis.supstat.com/categories.html\#animation-ref}{Galeria
      do autor},
  \item \href{http://www.r-bloggers.com/?s=animation}{Busca no R
      Bloggers}
  \end{itemize}

\end{frame}
