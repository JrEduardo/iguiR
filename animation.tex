\section{\texttt{animation}}

%----------------------------------------------------------------------

\subsection{Descrição}

\begin{frame}

  \texttt{animation} contém uma variedade de funções para produzir
  animações com o R. As animações podem ser produzidas em vários
  formatos: flash, gif, html, pdf e vídeos.

  \begin{itemize}
    \itemsep1pt\parskip0pt\parsep0pt
  \item Autores: Yihui Xie, Lijia Yu, Weicheng Zhu
  \item Lançamento: 11-Nov-2007
  \item Versão: 2.3
  \item URL:
    \url{http://cran.r-project.org/web/packages/animation/index.html},
    \url{http://yihui.name/animation/}
  \end{itemize}

\end{frame}

%----------------------------------------------------------------------

\subsection{Como usar}

\begin{frame}

  modelo característico.

\end{frame}

%----------------------------------------------------------------------

\subsection{Exemplos}

\begin{frame}

  links para executáveis e galerias.

  Algumas aplicações com o animation:
  \begin{itemize}
    \itemsep1pt\parskip0pt\parsep0pt
  \item
    \href{http://vis.supstat.com/categories.html\#animation-ref}{Galeria
      do autor},
  \item \href{http://www.r-bloggers.com/?s=animation}{Busca no R
      Bloggers},
  \end{itemize}

\end{frame}